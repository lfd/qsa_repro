\newlength\boxwidth\setlength{\boxwidth}{(\picwidth-\innerwidth)/2-\hsep}
\newlength\boxheight\setlength{\boxheight}{(\picheight-\vsep-\vsep)/3}
\algnewcommand{\LeftComment}[1]{\State \(\triangleright\) #1}
\definecolor{qiskitcol}{RGB}{112,48,160}
\definecolor{pennylanecol}{RGB}{112,173,71}
\definecolor{tfqcol}{RGB}{255,192,0}

\newcommand{\pwidth}{0.24\linewidth}
\newcommand{\pchcorr}{\hspace*{-0.9em}}
\newcommand{\algsize}{\fontsize{6.6}{7.6}\selectfont}
\newcommand{\gendeepq}{\pchcorr\begin{minipage}{\pwidth}
\renewcommand*\ttdefault{cmvtt} % Proportional typewriter font. How very cool.
    % \begin{algorithm}
%\label{alg:rl}
\algsize\begin{algorithmic}
    \LeftComment{\textbf{Quantum Deep Q-Learning}}%~\cite{mnih2013playing,van2016deep,OwenLockwood.2020}}
    \State Init replay buf $\mathbb{D}$, (target) VQC $\theta^{(-)}$
%\State Initialize VQC with weights $\theta$ 
%\State Initialize target-VQC with weights $\theta^{-}$ 
\For{$s\gets 0, \texttt{steps}$}
    \State Sample $\mathbb{B}=$
    \State \hspace*{2em}$(s_t, a_t, s_{t+1}, r_{t+1})\backslash\mathbb{D}$
    \ForAll{$b_{i} \in \mathbb{B}$}
        \LeftComment{Target}
        \State $y_i \gets \gamma \max\limits_{a'} Q(s_{i+1}, a'; \theta^-)$ 
        \State $q_i \gets Q(s_i, a_i; \theta)$ \Comment{Q-Value}
    \EndFor 
    \State $L(\theta) \gets (y - q)^2$ \Comment{Loss}
    \State Update $\theta$ (param shift rule~\cite{mitarai2018quantum, schuld2019evaluating})
    \If{$s\bmod{\texttt{update}} = 0$}
    \State $\theta^{-} \gets \theta$
    \EndIf 
\EndFor 
\end{algorithmic}
%\end{algorithm}
\end{minipage}}

%\begin{algorithm}
%\caption{Multi Query Optimization with QAOA}
%\label{alg:mqo}

\newcommand{\genqaoa}{\pchcorr\begin{minipage}{\pwidth}
\renewcommand*\ttdefault{cmvtt}
\algsize\begin{algorithmic}
    \LeftComment{\textbf{Multi Query Optimization}}
    \State Init MQO QUBO $qm$, circ.\ params $\beta$, $\gamma$
\State $c_{\texttt{lin}} \texttt{,} c_{\texttt{qdr}} \gets \texttt{IsingCoeffs}(qm)$
\State $hl \gets \texttt{\{\}}$
\For{$k \gets 0, p$}
    \State $hl \gets hl\frown\texttt{CostHam(}c_{\texttt{lin}}, c_{\texttt{qdr}}\texttt{)}$
    \State $hl \gets hl\frown\texttt{MixerHam()}$
\EndFor 
\State $qc \gets \texttt{buildQCircuit(}hl, \beta, \gamma$ \texttt{)}
\State Initialize classical optimizer $opt$
\While{$\lnot$converged}
    \State $\beta, \gamma \gets opt\texttt{.step(}qc, \beta, \gamma \texttt{)}$ 
\EndWhile 
\State $r \gets \texttt{sample(}qc, \beta, \gamma \texttt{)}$
\end{algorithmic}\end{minipage}}

\newcommand{\codefont}{\fontsize{3.5}{4}\selectfont}
\lstset{language=python,basicstyle=\codefont\ttfamily,numbers=none, 
  backgroundcolor={},framesep=0pt,framerule=0pt,xleftmargin=0pt,tabsize=1,
  showtabs=true,showspaces=false}

\newcommand{\lsttopcorr}{\vspace*{-0.75em}}
%%%%%%%%%%%%%%%%%%%%%%%%%%%%%%% Deep Q Learning %%%%%%%%%%%%%%%%%%%%%%%
%% Qiskit
\newsavebox\qiskitla\newsavebox\qiskitlb
\begin{lrbox}{\qiskitla}
\begin{minipage}{(\boxwidth-\hsep)/2}\lsttopcorr
\begin{lstlisting}
# define circuit
class VQC_Layer(Module):
  def __init__(self, n_qubits, n_layers, shots, device):
    self.circuit = QuantumCircuit(n_qubits)
    # input part
    for i, input in enumerate(input_params):
      self.circuit.rx(input, i)
    for i in range(n_layers):
      self.generate_layer(weight_params[i*n_qubits*2 : (i+1)*n_qubits*2])
    readout_op = ListOp([
      ~StateFn(pauli_op_list([('ZZII', 1.0)])) @ StateFn(self.circuit),
      ~StateFn(pauli_op_list([('IIZZ', 1.0)])) @ StateFn(self.circuit)])
    qnn = OpflowQNN(readout_op,
       input_params=input_params,
       weight_params=weight_params,
\end{lstlisting}\vspace*{-1em}\end{minipage}
\end{lrbox}
\begin{lrbox}{\qiskitlb}
\begin{minipage}{(\boxwidth-\hsep)/2}\lsttopcorr
\begin{lstlisting}
       quantum_instance=qi,
       gradient=Gradient()
    self.qnn = TorchConnector(qnn, initial_weights=torch.Tensor(np.zeros(n_qubits*n_layers*2)))
    
  def generate_layer(self, params):
    # variational part
    for i in range(self.n_qubits):
        self.circuit.ry(params[i*2], i)
        self.circuit.rz(params[i*2+1], i)
    # entangling part
    for i in range(self.n_qubits):
        self.circuit.cz(i, (i+1) % self.n_qubits)
 
    # Q-value calculation
  def forward(self, inputs):
    return self.qnn(inputs)
\end{lstlisting}\vspace*{-1em}\end{minipage}
\end{lrbox}

%% Pennylane
\newsavebox\pennyla\newsavebox\pennylb
\begin{lrbox}{\pennyla}
\begin{minipage}{(\boxwidth-\hsep)/2}\lsttopcorr
\begin{lstlisting}
# define circuit and Q-value calculation
@qml.qnode(device, wires=config.n_qubits), interface='tf', diff_method='parameter-shift')
def circuit(inputs, weights):
  # input part
  for i in range(config.n_qubits):
    qml.RX(inputs[i], wires=i)

  for i in range(config.n_layers):
    generate_layer(weights[i], config.n_qubits)
  return [qml.expval(qml.PauliZ(0) @ qml.PauliZ(1)), qml.expval(qml.PauliZ(2) @ qml.PauliZ(3))]
\end{lstlisting}\vspace*{-1em}\end{minipage}
\end{lrbox}
\begin{lrbox}{\pennylb}
\begin{minipage}{(\boxwidth-\hsep)/2}\lsttopcorr
\begin{lstlisting}
def generate_layer(params, n_qubits): 
  # variational part
  for i in range(n_qubits):
    qml.RY(params[i][0], wires=i)
    qml.RZ(params[i][1], wires=i)
  # entangling part
  for i in range(n_qubits):
    qml.CZ(wires=[i, (i+1) % n_qubits])

# initialize Keras Layer
VQC_Layer = qml.qnn.KerasLayer(
  qnode=circuit,
  weight_shapes={'weights': (config.n_layers, config.n_qubits, 2)},
  weight_specs = {"weights": {"initializer": "Zeros"}},
  output_dim=2,
  name='VQC_Layer')
\end{lstlisting}\vspace*{-1em}\end{minipage}
\end{lrbox}


%% TensorflowQuantum
\newsavebox\tqla\newsavebox\tqlb
\begin{lrbox}{\tqla}
\begin{minipage}{(\boxwidth-\hsep)/2}\lsttopcorr
\begin{lstlisting}
# define circuit
class VQC_Layer(keras.layers.Layer):
def __init__(self, n_qubits, n_layers):
  # ... definition of variables 
  circuit = cirq.Circuit()
  circuit.append([cirq.rx(inputs[i]).on(qubit) for i, qubit in enumerate(self.qubits)])
  for i in range(n_layers):
    circuit.append(
      self.generate_layer(params[i*n_qubits*2 : (i+1)*n_qubits*2]))
  readout_op = [
    cirq.PauliString(cirq.Z(qubit) for qubit in self.qubits[:2]),
    cirq.PauliString(cirq.Z(qubit) for qubit in self.qubits[2:])]
  self.vqc = tfq.layers.ControlledPQC(circuit, readout_op, 
    differentiator=ParameterShift())
\end{lstlisting}\vspace*{-1em}\end{minipage}
\end{lrbox}
\begin{lrbox}{\tqlb}
\begin{minipage}{(\boxwidth-\hsep)/2}\lsttopcorr
\begin{lstlisting}
def generate_layer(self, params):
  circuit = cirq.Circuit()
  # variational part
  for i, qubit in enumerate(self.qubits):
    circuit.append([
      cirq.ry(params[i*2]).on(qubit),
      cirq.rz(params[i*2+1]).on(qubit)])
  # entangling part
  for i in range(self.n_qubits):
    circuit.append(cirq.CZ.on(self.qubits[i], 
      self.qubits[(i+1) % self.n_qubits]))
  return circuit

# Q-value calculation
def call(self, inputs):
  # ... classical input processing
  return self.vqc([tiled_up_circuits, joined_vars])
\end{lstlisting}\vspace*{-1em}\end{minipage}
\end{lrbox}





%%%%%%%%%%%%%%%%%%%%%%%%%%%%%%% Multi-Query Optimisation %%%%%%%%%%%%%%%%%%%%%%%
%% Qiskit
\newsavebox\qiskitra\newsavebox\qiskitrb
\begin{lrbox}{\qiskitra}
\begin{minipage}{(\boxwidth-\hsep)/2}\lsttopcorr
\begin{lstlisting}
def construct_model(model,queries,
                    costs,savings):
  v = model.binary_var_list(len(costs))
  epsilon = 0.25
  wl= calculate_wl(costs, epsilon)
  wm= calculate_wm(savings, wl)
  El=model.sum(-1*(wl-costs[i])*v[i] \
     for i in range(0, len(costs)))
  Em=model.sum(model.sum(wm*v[i]*v[j] \
     for (i,j) in itertools.combinations
                  (queries[k], 2)) \
     for k in queries.keys())
  Es=model.sum(-s*v[i]*v[j] \
     for ((i,j), s) in savings)
  return(El + Em + Es)
\end{lstlisting}\vspace*{-1em}\end{minipage}
\end{lrbox}
\begin{lrbox}{\qiskitrb}
\begin{minipage}{(\boxwidth-\hsep)/2}\lsttopcorr
\begin{lstlisting}
def solve_with_QAOA(qubo):
  qmeas=qiskit.algorithms.QAOA(
    quantum_instance=Aer.get_backend('qasm_simulator'), 
    initial_point=[0., 0.])
  qaoa=MinimumEigenOptimizer(qaoa_meas)
  qres=qaoa.solve(qubo)
  return qres,qmeas.get_optimal_circuit()
	
def solve_MQO(queries, costs, savings):    
  model=Model('docplex_model')
  model.minimize(construct_model(
      model, queries, costs, savings))
  qubo=QuadraticProgram()
  qubo.from_docplex(model)
  result_QAOA, QAOA_circuit=solve_with_QAOA(qubo)
\end{lstlisting}\vspace*{-1em}\end{minipage}
\end{lrbox}

%% Pennylane
\newsavebox\pennyra\newsavebox\pennyrb
\begin{lrbox}{\pennyra}
\begin{minipage}{(\boxwidth-\hsep)/2}\lsttopcorr
\begin{lstlisting}
def get_ising_model(queries, costs,
                    savings):
  model = Model('docplex_model')
  v = model.binary_var_list(len(costs))
  epsilon = 0.25
  wl = calculate_wl(costs, epsilon)
  wm = calculate_wm(savings, wl)
  El = model.sum(-1*(wl-costs[i])*v[i] \
       for i in range(0, len(costs)))
  Em = model.sum(model.sum(wm*v[i]*v[j] \
       for (i,j) in itertools.combinations(
          queries[k], 2)) for k in queries.keys())
  Es = model.sum(-s*v[i]*v[j] \
       for ((i,j), s) in savings)
  model.minimize(El+Em+Es)  
  qubo = translators.from_docplex_mp(model)
  return qubo.to_ising()	
\end{lstlisting}\vspace*{-1em}\end{minipage}
\end{lrbox}
\begin{lrbox}{\pennyrb}
\begin{minipage}{(\boxwidth-\hsep)/2}\lsttopcorr
\begin{lstlisting}
def solve_MQO(queries, costs, savings, p=1):   
  sing, offset = get_ising_model(
                    queries, costs, savings)
  wires = range(len(costs))
  cham = create_cost_hamiltonian(linear, quadratic, offset)
  mham = create_mixer_hamiltonian(wires)
  dev = qml.device("default.qubit", wires=wires)
  circuit = qml.QNode(cost_function, dev)   
  params = get_initial_params(p)  
  optimizer = qml.GradientDescentOptimizer()
  steps = 200  
  for i in range(steps):
    params = optimizer.step(circuit, params,
                wires=wires, depth=p,
                cost_hamiltonian=cham,
                mixer_hamiltonian=mham)
\end{lstlisting}\vspace*{-1em}\end{minipage}
\end{lrbox}


%% TensorflowQuantum
\newsavebox\tqra\newsavebox\tqrb
\begin{lrbox}{\tqra}
\begin{minipage}{(\boxwidth-\hsep)/2}\lsttopcorr
\begin{lstlisting}
def get_ising_model(queries,costs,savings):
  model = Model('docplex_model')
  v = model.binary_var_list(len(costs))
  epsilon = 0.25
  wl = calculate_wl(costs, epsilon)
  wm = calculate_wm(savings, wl)
  El = model.sum(-1*(wl-costs[i])*v[i] for i in range(0, len(costs)))
  Em = model.sum(model.sum(wm*v[i]*v[j] \
            for (i,j) in itertools.combinations(queries[k], 2)) 
            for k in queries.keys())
  Es = model.sum(-s*v[i]*v[j] for ((i,j), s) in savings)
  model.minimize(El+Em+Es)  
  qubo = translators.from_docplex_mp(model)
  return qubo.to_ising()
def solve_MQO(queries, costs, savings, p=1):
  ising, offset = get_ising_model(queries, costs, savings)
\end{lstlisting}\vspace*{-1em}\end{minipage}
\end{lrbox}
\begin{lrbox}{\tqrb}
\begin{minipage}{(\boxwidth-\hsep)/2}\lsttopcorr
\begin{lstlisting}
  coeffs = np.real(ising.primitive.coeffs)
  pauli_array = ising.primitive.settings['data'].array
  linear, quadratic = get_coefficients_from_Pauli_array(pauli_array, coeffs)
  cirq_qubits = cirq.GridQubit.rect(1, len(costs))
  # ... Init parameters and hamiltonians ...
  qaoa_circuit = tfq.util.exponential(operators=hamiltonians,
                          coefficients=qaoa_parameters)
  hadamard_circuit = get_hadamard_circuit(cirq_qubits)
  # ... Initialize Keras Model ...
  model.add(tfq.layers.PQC(model_circuit,
            model_readout, backend=cirq.Simulator()))
  # ... Train the model ...
\end{lstlisting}\vspace*{-1em}\end{minipage}
\end{lrbox}
